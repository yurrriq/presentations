\documentclass[english]{beamer}
%% \usetheme{Boadilla}
\usetheme{Madrid}
\usecolortheme{seahorse}
\setbeamerfont{footnote}{size=\tiny}
\setbeameroption{show notes on second screen=right}

\usepackage{dirtytalk}
\usepackage{emoji}
%% \usepackage{epstopdf}
\newcommand{\yep}{\emoji{check-mark-button}}
\newcommand{\kinda}{\emoji{warning}}
\newcommand{\nope}{\emoji{cross-mark}}
\usepackage{hyperref}
\newcommand{\hrefootnote}[2]{\href{#1}{#2}\footnote{\url{#1}}}
\usepackage{minted}
\usemintedstyle{friendly}
\usepackage{qrcode}
\newcommand{\supasterisk}{\textsuperscript{*}}
%% \newcommand{\supdagger}{\textsuperscript{\textdagger}}
%% \newcommand{\supddagger}{\textsuperscript{\textddagger}}

\title{EKS Auto Mode: Behind the Curtain}
\author[Eric Bailey]{Eric Bailey}
\date[Kubernetes Slovenia 16.04.2025]{Kubernetes Slovenia \\ 16th April 2025}

\begin{document}

\begin{frame}
  \titlepage
  \note{
    Briefly mention how it's presented as a turnkey or magic solution to the
    problem of managing Kubernetes, and that we're going to ``look behind the
    curtain.''
  }
\end{frame}

\begin{frame}
  \frametitle{Table of Contents}
  \tableofcontents
\end{frame}

\section{Introduction}
\begin{frame}
  \frametitle{Introduction}
  \begin{itemize}
    \item What is EKS Auto Mode?
          \note[item]<1->{
            It's basically vanilla EKS with some common add-ons, including
            highly opinionated or restrictive installations of the AWS VPC CNI
            and Karpenter.
          }
          \pause \\
          \say{%
            Amazon EKS Auto Mode fully automates Kubernetes cluster management for
            compute, storage, and networking on AWS with a single click.%
          }%% --- \href{https://aws.amazon.com/eks/auto-mode/}{Amazon EKS Auto Mode}%
          \footnote<2->{\url{https://aws.amazon.com/eks/auto-mode/}}
          \pause
    \item What is EKS?
          \note[item]<3->{
            I'll refer to this as ``vanilla EKS'' to differentiate from EKS
            Auto Mode.
          }
          \note[item]<3->{
            It's ``Kubernetes as a Service'', i.e., AWS-managed control plane
            and a small collection of official add-ons, as well as a broader
            add-on marketplace.
          }
          \note[item]<3->{
            What's covered by AWS Support is sensibly minimal, and it's slightly
            prescriptive, e.g., feature flags, blessed CNIs, etc.
          }
          \pause \\
          \say{%
            Amazon Elastic Kubernetes Service (Amazon EKS) is the premiere platform
            for running Kubernetes clusters, both in the Amazon Web Services (AWS)
            cloud and in your own data centers (EKS Anywhere and Amazon EKS Hybrid
            Nodes).%
          }%% --- \href{https://docs.aws.amazon.com/eks/latest/userguide/what-is-eks.html}{What is Amazon EKS?}%
          \footnote<4->{\url{https://docs.aws.amazon.com/eks/latest/userguide/what-is-eks.html}}
          \pause
    \item What is kOps? \pause \\
          \say{%
            The easiest way to get a production grade Kubernetes cluster up and
            running.%
          }%% --- \href{https://kops.sigs.k8s.io/}{kOps - Kubernetes Operations}%
          \footnote<6->{\url{https://kops.sigs.k8s.io/}}
  \end{itemize}
\end{frame}

\section{kOps - Kubernetes Operations}
\begin{frame}[fragile]
  \frametitle{kOps - Kubernetes Operations}

  kOps provides several managed add-ons, configurable through the cluster spec,
  as well as the means to specify custom static add-ons.%
  \footnote{\url{https://kops.sigs.k8s.io/addons/}}
  \vskip 1em

  \pause
  For example, \href{https://github.com/kubernetes/autoscaler/tree/master/cluster-autoscaler}{Cluster Autoscaler}%
  \footnote<2->{\url{https://github.com/kubernetes/autoscaler/tree/master/cluster-autoscaler}}%
  can be configured as follows.
  \begin{minted}[fontsize=\scriptsize,frame=lines]{yaml}
    spec:
      clusterAutoscaler:
        enabled: true
        # ...
  \end{minted}
  \vskip 1em

  \pause
  Managed add-ons will be upgraded following the kOps and Kubernetes lifecycle,
  and are configured based on the \href{https://kops.sigs.k8s.io/cluster_spec/}{cluster spec}%
  \footnote<3->{\url{https://kops.sigs.k8s.io/cluster_spec/}}.
\end{frame}

\begin{frame}
  \frametitle{kOps ``Auto Mode''}

  Using kOps built-in features and managed add-ons, it's easy to achieve a robust
  Kubernetes cluster, with minimal TCO. \pause

  \begin{itemize}
    \item CNI\footnote{\url{https://kops.sigs.k8s.io/networking/\#cni}}
          \note[item]<2->{kOps supports several CNIs, such as the AWS VPC CNI and Cilium.}
    \item CoreDNS
    \item kube-proxy \pause
    \item Node-local DNS cache \pause
    \item Cluster Autoscaler (and NTH) \pause or Karpenter \pause
    \item AWS EBS CSI driver \pause (and snapshot controller) \pause
    \item Metrics Server \pause
    \item Pod Identity Webhook \pause
    \item AWS Load Balancer Controller \pause
    \item cert-manager
          \note[item]<11->{Mention hen and egg for some kOps components such as Metrics Server.}
  \end{itemize}
\end{frame}

\section{EKS}

\begin{frame}
  \frametitle{EKS ``Manual Mode''}

  For EKS there exist several ``managed''\footnote{Sane defaults, optionally
    managed lifecycle, backed by E2E testing by Amazon.} add-ons, tens of
  community\supasterisk and marketplace add-ons\note[item]{You can also write
    your own add-ons.}, and
  \hrefootnote{https://aws-ia.github.io/terraform-aws-eks-blueprints/}{Amazon
    EKS Blueprints for Terraform} can help fill the remaining gaps.
  \note[item]{There be dragons. Managing Helm charts via Terraform
    is\ldots\ interesting.} \pause

  \begin{itemize}
    \item AWS VPC CNI
    \item CoreDNS
          \note[item]<2->{Mention how node-local DNS cache is just a special case.}
          \note[item]<2->{Not possible to configure two instances of CoreDNS add-on.}
    \item kube-proxy \pause
    \item No autoscaler add-on\ldots\pause\ but MNGs play nice with CAS \emoji{slightly-frowning-face} \pause
    \item EBS CSI driver \pause (and snapshot controller) \pause
    \item Metrics Server\supasterisk
          \note[item]<7->{Metrics Server and cert-manager are community add-ons.}
          \pause
    \item Pod Identity Webhook \pause \emoji{magic-wand} \pause
          \note[item]<8->{Pod identity webhook ``for free''.}
    \item AWS Load Balancer Controller \pause \emoji{confounded-face}
          \note[item]<10->{Need to DIY LBC.}
          \pause
    \item cert-manager\supasterisk
  \end{itemize}
\end{frame}

\section{EKS Auto Mode}
\begin{frame}
  \frametitle{EKS Auto Mode}

  Out of the box\footnote{\ldots assuming the right options are specified.}, EKS Auto Mode
  supports the following
  \hrefootnote{https://docs.aws.amazon.com/eks/latest/userguide/automode.html\#_automated_components}{automated
    components}.
  \pause

  \begin{itemize}
    \item AWS VPC CNI \pause (with caveats)
      \note[item]<3->{EKS Auto Mode does not support warm IP, warm prefix, or warm ENI configurations.}
      \note[item]<3->{Instead of ds/aws-node it's implemented as a systemd unit that can't be reconfigured.}
      \note[item]<3->{tl;dr custom networking is not supported.}
          \pause
    \item Karpenter \pause (with default and custom NodePools and NodeClasses)
      \note[item]<4->{Auto Mode comes with NodePools general-purpose and platform (tainted \texttt{CriticalAddonsOnly:NoSchedule}).}
      \note[item]<4->{Default NodePools dont' support SSH and such.}
      \note[item]<4->{You can add your own NodePools, but probably don't wanna have different CNI configs.}
          \pause
    \item CoreDNS \pause
          \note[item]<5->{You can configure this, e.g., to use a node-local DNS cache, but you'll probably have to install it off the shelf first, and then reconfigure, then install node-local DNS, then reconfigure\ldots}
    \item Security patches, OS and component upgrades
          \note[item]<7->{\say{EKS Auto Mode enforces a 21-day maximum node
              lifetime to ensure up-to-date software and APIs.}} \pause
    \item Pod Identity Webhook \pause
    \item AWS Load Balancer controller%
          \footnote<9->{There is no migration path for existing load balancers.}
  \end{itemize}
\end{frame}

\section{So, which approach is best?}
\begin{frame}
  \frametitle{So, which approach is best?}

  \pause
  \begin{center}
    It depends.\textsuperscript{\texttrademark}
  \end{center}
\end{frame}

\begin{frame}[plain]
  \begin{center}
    \Huge Thank you! \emoji{person-bowing} \\
  \end{center}
\end{frame}

\begin{frame}
  \frametitle{But actually\ldots}

  Here are some criteria to rule out EKS Auto Mode as a viable option.

  \begin{itemize}
    \item Need custom networking?
          \begin{itemize}
            \pause
            \item \emoji{face-with-head-bandage}
                  \say{%
                    The IP Addresses of Pods and Nodes must be from the same
                    CIDR Block.%
                  }\footnote<2->{The documentation is misleading and
                    contradictory, YMMV.}
                  \pause
            \item \emoji{money-with-wings} EKS Auto Mode does not support
                  warm IP, warm prefix, or warm ENI configurations.
                  \note[item]<3->{Mention our IPAM cost drivers deep dive.}
                  \pause
            \item \emoji{grimacing} Want to use a node-local DNS cache?
                  \note[item]<4->{Custom configuration is supported, but likely
                    would need to be added after initial cluster provisioning.}
          \end{itemize}
          \pause
    \item Have an existing EKS cluster?
          \begin{itemize}
            \pause
            \item \emoji{chart-decreasing} Can't tolerate downtime for
                  existing load balancers?
          \end{itemize}
  \end{itemize}
\end{frame}

\begin{frame}
  \frametitle{Feature Comparsion}

  Comparing the high-level
  \hrefootnote{https://docs.aws.amazon.com/eks/latest/userguide/automode.html\#_features}{features}

  \begin{tabular}{ r|c c c }
                   & EKS Auto Mode & EKS    & kOps   \\
    \hline
    Nodes          & \kinda        & \yep   & \yep   \\
    Autoscaling    & \yep          & \kinda & \yep   \\
    Upgrades       & \yep          & \kinda & \kinda \\
    Load balancing & \kinda        & \yep   & \yep   \\
    Storage        & \yep          & \yep   & \yep   \\
    Networking     & \nope         & \kinda & \yep   \\
    IAM            & \yep          & \yep   & \yep   \\
    AMI/OS         & \nope         & \yep   & \yep
  \end{tabular}
  \note[item]{EKS Auto Mode uses their custom Bottlerocket AMI.}
  \note[item]{I haven't tested GPU workloads. They're supposed to work, but NVIDIA is gonna NVIDIA\ldots}
  \note[item]{%
    On EKS and kOps you can use pretty much anything, so long as you have a
    compatible kubelet, but Karpenter is getting increasingly prescriptive,
    which is not inherently bad.%
  }
\end{frame}

\begin{frame}[plain]
  \begin{center}
    Slides available on GitHub: \\
    \vskip 2em
    \qrcode{https://github.com/yurrriq/presentations}
  \end{center}
\end{frame}

\end{document}
